\documentclass[12pt,a5paper,ngerman,titlepage]{article}
\usepackage[T1]{fontenc}
\usepackage[utf8]{inputenc}
\usepackage{hyperref}
\usepackage{babel}
\usepackage[paper=a5paper,left=15mm,right=15mm,top=20mm,bottom=20mm]{geometry}
\usepackage{amsmath}
\usepackage{amsfonts}
\usepackage{circuitikz}
\usepackage[headsepline=1pt,automark]{scrlayer-scrpage}
\automark{section}
\clearpairofpagestyles
\ihead{\leftmark}
\ohead{\pagemark}
\setkomafont{pagehead}{\footnotesize}% normale Schrift in Kopfzeile
\renewcommand*\familydefault{\sfdefault}



\title{\Huge \textbf{Formelsammlung} \\ \large für die \\HB3/9 Prüfung}
\author{\textbf{Charpoan Kong HB9HJN}}
\date{}
 
\begin{document}
\maketitle
\pagenumbering{Roman}
\tableofcontents
\newpage
\pagenumbering{arabic}% Arabic page numbers (and reset to 1)
\section{SI-Vorzeichen \& Einheiten}
\begin{table}[!h]
\begin{tabular}{ | l | l | l | l |}
\hline
T 		& Tera 	& $10^{12}$	 & 1000000000000	\\ \hline
G 		& Giga 	& $10^9$ 	 & 1000000000 		\\ \hline
M 		& Mega 	& $10^6$ 	 & 1000000 			\\ \hline
k 		& Kilo 	& $10^3$ 	 & 1000	 			\\ \hline
m 		& Milli & $10^{-3}$  & 0.001	 		\\ \hline
$\mu$ 	& Mikro & $10^{-6}$  & 0.000001	 		\\ \hline
n 		& Nano 	& $10^{-9}$  & 0.000000001	 	\\ \hline
p 		& Pico 	& $10^{-12}$ & 0.000000000001	\\ \hline

\end{tabular}
\end{table}
\begin{table}[!h]
\begin{tabular}{ | l | l | l | l |}
\hline
Ladung 				& Q 	& Coulomb 			& $C=As$				 \\ \hline
Spannung 			& U 	& Volt 	 			& $V$ 					 \\ \hline
Leistung 			& L 	& Watt 	 			& $W$ 					 \\ \hline
Arbeit 				& W 	& Wattsekunde 	 	& $VAs$	 				 \\ \hline
Impedanz 			& R	 	& Ohm  				& $\Omega=\cfrac{V}{A}$	 \\ \hline
Leitwert 			& G 	& Siemens  			& $S = \cfrac{1}{\Omega}$ \\ \hline
Kapazität 			& C 	& Farad  			& $F = \cfrac{As}{V}$	 \\ \hline
Induktivität		& L 	& Henry 			& $H = \cfrac{Vs}{A}$	 \\ \hline
El. Feldstärke		& E 	& Volt pro Meter 	& $\cfrac{V}{m}$			 \\ \hline
Mag. Feldstärke		& H 	& Ampere pro Meter 	& $\cfrac{A}{m}$			 \\ \hline
Flussdichte			& B 	& Tesla 			& $T = \cfrac{Vs}{m^2}$	 \\ \hline
Frequenz			& $f$ 	& Herz 				& $Hz = \cfrac{1}{s}$	 \\ \hline
\end{tabular}
\end{table}
\newpage
\section{Ohmisches/Leistungs Dreieck \& Wellenlänge}
\subsection{Spannung}

$$U=RI =\frac{P}{I} = \sqrt{PR}$$

\subsection{Strom}
$$I=\frac{P}{U}=\frac{U}{R}=\sqrt{\frac{P}{R}} $$

\subsection{Wiederstand}
$$R=\frac{U}{I}=\frac{P}{I^2}=\frac{U^2}{P} $$

\subsection{Leistung}
$$P=UI =\frac{U^2}{R} =RI^2$$
\subsection{Wellenlänge \& Frequenz}
\[
\begin{minipage}{.5\linewidth}
  $\lambda=\cfrac{c}{f}$ \\[5pt]
  $f=\cfrac{c}{\lambda}$
\end{minipage}%
\begin{minipage}{.5\linewidth}
  \fontsize{8pt}{9pt}\selectfont
  $c = Lichtgeschwindugkeit \approx 3*10^{8}$ \\
  $c = 2.99792458*10^{8}$

\end{minipage}
\]
\newpage
\section{Widerstand \& Leistung}
\subsection{Serieschaltung}
$$R_{\sum} = \sum R_{i}$$

\subsection{Paralellschaltung}
$$R_{\sum} = \frac{1}{\sum \frac{1}{R_{i}}}$$

\subsection{Leiterwiderderstand}
\[
\begin{minipage}{.5\linewidth}
  $R=\cfrac{\rho l}{A}$ 
\end{minipage}%
\begin{minipage}{.5\linewidth}
  \fontsize{8pt}{9pt}\selectfont
  $\rho = spezifischer \; Widerstand$

\end{minipage}
\]
\subsection{Spannungsteiler}
$$U_{x}=R_{x}\frac{U}{R_{ges}}$$

\subsection{Wirkungsgrad}
\[
\begin{minipage}{.5\linewidth}
  $\eta = \cfrac{P_{out}}{P_{in}}$ \\[5pt]
  $P_{in} = P_{out}+P_{V}$
\end{minipage}%
\begin{minipage}{.5\linewidth}
  \fontsize{8pt}{9pt}\selectfont
  $P_{V} = Verlustleistung$

\end{minipage}
\]
\newpage
\section{Wechselstrom}
\subsection{Effektivspannung}
\subsubsection{Sinus}
$$U_{eff}=\cfrac{\text{\^{U}}}{\sqrt{2}}$$
\subsubsection{Dreieck}
$$U_{eff}=\cfrac{\text{\^{U}}}{\sqrt{3}}$$
\subsubsection{Rechteck}
$$U_{eff}=\text{\^{U}}\sqrt{DutyCycle}$$
\newpage
\section{Kondensator}
\subsection{Kapazität}
\[
\begin{minipage}{.5\linewidth}
	
  $C = \varepsilon_{0} \varepsilon_{r} \cfrac{A}{d}$
\end{minipage}%
\begin{minipage}{.5\linewidth}
  \fontsize{8pt}{9pt}\selectfont
  $\varepsilon_{0} = Elektrische \; Feldkonstante$ \\
  $\varepsilon_{r} = Permittivit\ddot{a}t$ \\
  $\varepsilon_{0} = 8.854187817*10^{-12}$

\end{minipage}
\]

\subsection{Serieschaltung}
$$C_{\sum} = \frac{1}{\sum \frac{1}{C_{i}}}$$

\subsection{Paralellschaltung}
$$C_{\sum} = \sum C_{i}$$

\subsection{$\tau$/Zeitkonstante}
\begin{align*}
&\tau = RC\\[5pt]
&\lim_{U \to 0\%/100\%}\Delta t = 5\tau 
\end{align*}

\subsection{Dreh-/Plattenkondensator}
\[
\begin{minipage}{.5\linewidth}
  \centering
  $C_{p}=\cfrac{f_{u}^2 \Delta C}{f_{o}^2-f_{u}^2}-C_{a}$
\end{minipage}%
\begin{minipage}{.5\linewidth}
\fontsize{8pt}{9pt}\selectfont
  $C_{p} = Paralellkapazit\ddot{a}t$ \\
  $C_{a} = Anfangskapazit\ddot{a}t$ \\
  $f_{u} = untere \; Frequenz$ \\
  $f_{o} = obere \; Frequenz$ \\
  $\Delta C = Kapazit\ddot{a}t \; des \; Drehko$ 
\end{minipage}
\]
\subsection{Kapazitiver Blindwiederstand}

\begin{align*}
&X_{c} = \frac{U}{I} = \frac{1}{\omega C} = \frac{1}{2\pi fC} \\[5pt]
&C=\frac{1}{\omega X_{c}} = \frac{1}{2\pi fX_{c}} \\[5pt]
&f=\frac{1}{2\pi X_{c}C} \\[5pt]
&I =\frac{U}{X_{c}}
\end{align*}
\subsection{Verlustfaktor/Güte}
\[
\begin{minipage}{.5\linewidth}
	
  $\tan \delta = \cfrac{I_{R}}{I_{c}} = \cfrac{X_{c}}{R_{p}}$\\[5pt]
  $Q = \cfrac{R_{p}}{X_{c}}$
\end{minipage}%
\begin{minipage}{.5\linewidth}
  \fontsize{8pt}{9pt}\selectfont
  $R_{p} = paraleller \; Verlustwiederstand$ \\
  $I_{R} = Strom \; durch \; R_{v}$ \\
  $I_{C} = Strom \; durch \; Kondensator$ \\

\end{minipage}
\]
\newpage
\section{Spule}
\subsection{Induktivität}
\[
\begin{minipage}{.5\linewidth}
  $L=\cfrac{\mu_{0}\mu_{r}AN^2}{l}=A_{L}N^2$ \\[5pt]
  $A_{L}=\cfrac{\mu_{0}\mu_{r}A}{l}$
\end{minipage}%
\begin{minipage}{.5\linewidth}
  \fontsize{8pt}{9pt}\selectfont
  $\mu_{0} = Permeabilit\ddot{a}t \;im \; luftleeren \; Raum$ \\
  $\mu_{r} = Permeabilit\ddot{a}t \; des \; Kernmaterials$ \\
  $A_{L} = Wert \; vorgefertigter \; Kerne$ \\

\end{minipage}
\]
\subsection{Induktion- \& Selbstinduktionspannung}
\begin{align*}
&U_{ind}=-L\frac{\Delta I}{\Delta t} \\[5pt]
&L = -U_{ind}\frac{\Delta t}{\Delta I}
\end{align*}

\subsection{Serieschaltung}
$$L_{\sum} = \sum L_{i}$$

\subsection{Paralellschaltung}
$$L_{\sum} = \frac{1}{\sum \frac{1}{L_{i}}}$$

\subsection{$\tau$/Zeitkonstante}

$$\tau = \frac{L}{R}$$


\subsection{Verlustfaktor/Güte}
\[
\begin{minipage}{.5\linewidth}
	
  $\tan \delta = \cfrac{I_{R}}{I_{L}} = \cfrac{R_{s}}{X_{L}}$\\[5pt]
  $Q = \cfrac{X_{L}}{R_{s}}$
\end{minipage}%
\begin{minipage}{.5\linewidth}
  \fontsize{8pt}{9pt}\selectfont
  $R_{s} = serielle \; Verlustwiederstand$ \\
  $I_{R} = Strom \; durch \; R_{v}$ \\
  $I_{L} = Strom \; durch \; Spule$ \\

\end{minipage}
\]
\subsection{Induktiver Blindwiederstand}
\begin{align*}
&X_{L} = \omega L=2\pi fL \\[5pt]
&L = \frac{X_{L}}{\omega} = \frac{X_{L}}{2\pi f} \\[5pt]
&f = \frac{X_{L}}{2\pi L}
\end{align*}
\newpage
\section{Impedanz}
\subsection{Serieschaltung}
$$Z=\sqrt{R^2+(X_{L}-X_{C})^2}$$

\subsection{Paralellschaltung}
$$Z=\sqrt{\frac{1}{R}^2+\left(\frac{1}{X_{L}}-\frac{1}{X_{C}}\right)^2}$$
\newpage
\section{Transformator/Übertrager}
\subsection{Spannungs-/Strom-/Windungs-/ Wiederstandsübersetzung}
\begin{align*}
&\ddot{u}=\cfrac{U_{1}}{U_{2}}=\cfrac{N_{1}}{N_{2}}=\cfrac{I_{2}}{I_{1}} =\sqrt{\cfrac{Z_{1}}{Z_{2}}} \\[5pt]
&I_{1}=I_{2}\cfrac{U_{2}}{U_{1}} = I_{2}\cfrac{N_{2}}{N_{1}} =I_{2}\sqrt{\cfrac{Z_{2}}{Z_{1}}}\\[5pt]
&I_{2}=I_{1}\cfrac{U_{1}}{U_{2}} = I_{1}\cfrac{N_{1}}{N_{2}} =I_{1}\sqrt{\cfrac{Z_{1}}{Z_{2}}}\\[5pt]
\end{align*}
\subsection{Stromdichte}
$$S=\frac{I}{A}$$
\newpage
\section{RC-Glied}
\subsection{Grenzfrequenz}
\begin{align*}
&f_{g}=\frac{1}{2\pi RC} \\[5pt]
&C =\frac{1}{2\pi f_{g}R} \\[5pt]
&R =\frac{1}{2\pi f_{g}C}
\end{align*}

\subsection{Shape-Faktor}
$$ShapeFaktor=\frac{Bandbreite \; bei \;60db}{Bandbreite \; bei \; 6db}$$
\newpage
\section{Dezibel}
\subsection{Dezibel bei Leistug}
$$\nu=10\log{\left(\frac{P_{out}}{P_{in}}\right)}$$
\subsection{Dezibel bei Spannung}
$$\nu=20\log{\left(\frac{U_{out}}{U_{in}}\right)}$$
\newpage
\section{LC-Schwingkreis}
\subsection{Resonanzfrequenz}
$$f_{res} = \frac{1}{2\pi \sqrt{LC}} = \frac{f_{max}+f_{min}}{2}$$
\begin{align*}
&L = \frac{1}{(2\pi f)^2C} \\[5pt]
&C = \frac{1}{(2\pi f)^2L} \\[5pt]
\end{align*}
 
\subsection{Bandbreite}
$$b = f_{max}-f_{min}= \frac{f_{res}}{Q}$$
\subsection{Güte}
\[
\begin{minipage}{.5\linewidth}
	
  $Q = \cfrac{1}{R_{s}}*\sqrt{\cfrac{L}{C}} = \cfrac{f_{res}}{b}= \cfrac{R_{p}}{X_{L}} =$ \\[5pt]
  $ \cfrac{X_{L}}{R_{s}} $ \\[5pt]
  $b = \cfrac{R_{s}}{2\pi L}$\\[5pt]
  $R_{s} = \cfrac{1}{Q}*\sqrt{\cfrac{L}{C}}$\\[5pt]
  $R_{res} = \cfrac{2 \pi f_{res} L}{Q}$

\end{minipage}%
\begin{minipage}{.5\linewidth}
  \fontsize{8pt}{9pt}\selectfont
  $R_{s} = serieller \; Verlustwiederstand $ \\
  $R_{res} = Resonanz \; Verlustwiederstand $ \\
  $R_{p} = paraleller \; Verlustwiederstand $ \\


\end{minipage}
\]
\newpage
\section{Diode}
\subsection{Vorwiderstand}
$$R=\frac{U_{cc}-U_{F}}{I_{F}}$$
\subsection{Spannungsfestigkeit/Max. Spannung}
$$U=U_{in}*\sqrt{2}_{oder \; anderer \;Faktor \; Spitzenspannung}$$
\newpage
\section{Transistor/FET}


\begin{center}
\begin{circuitikz}[scale=0.75,transform shape] \draw
(0,6) to[ short, o-o ] (6,6)
(0,0) node[anchor=east] {$-$}
(0,6) node[anchor=east] {$+$}
(0,3) node[anchor=east] {$in$}
(6,3.5) node[anchor=west] {$out$}
(2,0) to[ R, l_=$R_2$, *- ] (2,3)
(2,3) to[ R, l_=$R_1$, *-* ] (2,6)
(0,3) to[ C, l_=$C_1$, o- ] (2,3)

(4,6) to[ R,i_=$I_{C}$, l_=$R_C$, *- ] (4,3.5)
(4,3.5) to[ C, l_=$C_2$, *-o ] (6,3.5)
(4,2) to[ R, l_=$R_E$, *-* ] (4,0)
(5,0) to[ C, l_=$C_E$, *- ] (5,2)
(4,2) to[ short, *- ] (5,2) 
(4,2)  to[Tnpn] (4,4)
(2,3) to[ short,i_=$I_{B}$, *- ] (3.5,3)
(0,0) to[ short, o-o ] (6,0); 
\end{circuitikz}
\end{center}

\subsection{Stromverstärkungsfaktor}
$$\beta = \frac{I_{C}}{I_{B}}$$

\subsection{$R_{1}$}
\begin{align*}
& I_{B} = \cfrac{I_{E}}{\beta + 1} \\[5pt]
& I_{R_{1}} = 11 * I_{B}\\[5pt]
& U_{R_{1}} = U - U_{BE}\\[5pt]
& R_{1} = \cfrac{U_{R_{1}}}{I_{R_{1}}}\\[5pt]
\end{align*}

\subsection{$R_{C}$}
\begin{align*}
& I_{B} = \cfrac{I_{2}}{9}\\[5pt]
& I_{C} = I_{B}\; \beta \\[5pt]
& U_{R_{C}}=U-U_{C}\\[5pt]
& R_{C} = \cfrac{U_{R_{C}}}{I_{c}}\\[5pt]
\end{align*}

\subsection{$I_{C}$}
\begin{align*}
& I_{E} = \cfrac{U_{E}}{R_{E}}\\[5pt]
& I_{B} = \cfrac{I_{E}}{\beta + 1}\\[5pt]
& I_{C} = I_{B}\; \beta \\[5pt]
\end{align*}

\subsection{$P_{V}$}
\begin{align*}
& U_{R_{C}} = R_{C} \; I_{C}\\[5pt]
& U_{Transistor} = U-U_{R_{C}}\\[5pt]
& P_{Verlust} = U_{Transistor} * I_{C}\\[5pt]
\end{align*}

\newpage
\section{Operationsverstärker}
\subsection{Invertierender Verstärker}
\begin{center}
\begin{circuitikz}[scale=0.75,transform shape] \draw
(5,2) node[op amp] (opamp) {}
(0,2.5) to[ R, l_=$R_1$, o- ] (opamp.-)
(3,4) to[ short, -* ] (3,2.5)
(7,4) to[ short, -* ] (7,2)
(opamp.out) to[ short, - ] (8,2)
(0,2.5) node[anchor=east] {$in$}
(8,2) node[anchor=west] {$out$}
(opamp.+) node[ground] {}
(3,4) to[ R, l_=$R_2$, - ] (7,4);
\end{circuitikz}
\end{center}

$$U_{out}=-U_{in}\frac{R_{2}}{R_{1}}$$

\subsection{Nichtnvertierender Verstärker}
\begin{center}
\begin{circuitikz}[scale=0.75,transform shape] \draw
(5,2) node[op amp] (opamp) {}
(0,2.5) to[ R, l_=$R_1$, - ] (opamp.-)
(0,2.5) node[ground] {}
(3,4) to[ short, -* ] (3,2.5)
(7,4) to[ short, -* ] (7,2)
(opamp.out) to[ short, - ] (8,2)
(0,1.5) node[anchor=east] {$in$}
(8,2) node[anchor=west] {$out$}
(0,1.5) to[ short, o- ] (opamp.+)
(3,4) to[ R, l_=$R_2$, - ] (7,4);
\end{circuitikz}
\end{center}
$$U_{out}=1+\frac{R_{2}}{R_{1}}$$

\newpage
\subsection{Differenzialverstärker}
\begin{center}
\begin{circuitikz}[scale=0.75,transform shape] \draw
(5,4) node[op amp] (opamp) {}
(0,4.5) to[ R, l_=$R_1$, o- ] (opamp.-)
(3,6) to[ short, -* ] (3,4.5)
(7,6) to[ short, -* ] (7,4)
(opamp.out) to[ short, - ] (8,4)
(0,4.5) node[anchor=east] {$in_1$}
(0,3.5) node[anchor=east] {$in_2$}
(8,4) node[anchor=west] {$out$}
(0,3.5) to[ R, l_=$R_2$, o- ] (opamp.+)
(3,3.5) to[ R, l_=$R_4$, *- ] (3,1)
(3,1) node[ground] {}
(3,6) to[ R, l_=$R_3$, - ] (7,6);
\end{circuitikz}
\end{center}
\begin{align*}
&\nu_{U1} = \cfrac{R_{3}}{R_{1}} \\[5pt]
&\nu_{U2} = \cfrac{1+ \frac{R_{3}}{R_{1}}}{1+ \frac{R_{2}}{R_{4}}} \\[5pt]
&U_{out} = U_{in2}*\nu_{U2}-U_{in1}*\nu_{U1} \\[5pt]
\end{align*}
\newpage
\section{Elektromagnetisches Feld}

\subsection{Elektrische Feldstärke}
\begin{align*}
&E=\cfrac{U}{d} \\[5pt]
&\cfrac{E_{1}}{E_{2}}=\cfrac{d_{2}}{d_{1}} 
\end{align*}

\subsection{Magnetische Feldstärke}
$$H=\cfrac{I}{d}$$

\subsection{Magnetische Flussdichte}
\[
\begin{minipage}{.5\linewidth}
	
  $B= \mu_{0} \mu_{r} H$

\end{minipage}%
\begin{minipage}{.5\linewidth}
  \fontsize{8pt}{9pt}\selectfont
  $\mu_{0} = Permeabili\ddot{a}t \; 4\pi *10^{-7}\cfrac{Vs}{Am}$ \\
  $\mu_{r} = Permeabili\ddot{a}t \; des \; Materials $ \\


\end{minipage}
\]

\subsection{Strahlungsdichte Kuglestrahler}
\[
\begin{minipage}{.5\linewidth}
	
  $S=\cfrac{P_{ERP}}{4\pi r^2}$

\end{minipage}%
\begin{minipage}{.5\linewidth}
  \fontsize{8pt}{9pt}\selectfont
  $P_{ERP} = Leistung \; isotroper \; Strahler$
\end{minipage}
\]

\subsection{Feldwellenwiederstand}
\[
\begin{minipage}{.5\linewidth}
	
  $Z_{0}=\cfrac{E}{H}=\sqrt{\cfrac{\mu_{0}}{\varepsilon_{0}}}=120\pi \Omega$

\end{minipage}%
\begin{minipage}{.5\linewidth}
  \fontsize{8pt}{9pt}\selectfont
  $Z_{0} = Feldwellenwiederstand$
\end{minipage}
\]

\subsection{Ersatzfeldstärke}
\subsubsection{Allgemein}
\[
\begin{minipage}{.5\linewidth}
	
  $E=\cfrac{\sqrt{30 \Omega \: P_{ERIP}}}{r}$\\[5pt]
  $E=\cfrac{1}{r}\sqrt{\cfrac{Z_{0}}{4 \pi} \; P_{ERIP}}$\\[5pt]

\end{minipage}%
\begin{minipage}{.5\linewidth}
  \fontsize{8pt}{9pt}\selectfont
  $P_{ERIP} = Leistung \; isotroper \; Strahler$
\end{minipage}
\]

\subsubsection{Dipol}
$$E \approx 7\cfrac{\sqrt{P}}{r}$$

\subsection{Brauchbare Grenzfrequenz}
\[
\begin{minipage}{.5\linewidth}
	
  $MUF \approx \cfrac{f_{k}}{\sin \alpha}$ 

\end{minipage}%
\begin{minipage}{.5\linewidth}
  \fontsize{8pt}{9pt}\selectfont
  $MUF = maximum \; usable \; frequency$ \\
  $f_{k} = kritische \; Frequenz$
\end{minipage}
\]

\subsection{Optimale Grenzfrequenz}
\[
\begin{minipage}{.5\linewidth}
	
  $f_{opt} \approx 0.85 \; MUF$ 

\end{minipage}%
\begin{minipage}{.5\linewidth}
  \fontsize{8pt}{9pt}\selectfont
  $MUF = maximum \; usable \; frequency$ \\
  $f_{opt} = optimale \; Frequenz$
\end{minipage}
\]

\newpage
\section{Antennentechnik}

\subsection{Dipol}
\subsubsection{Länge}
$$ l=n \; \cfrac{\lambda}{2}  \quad n \in \mathbb{N} $$

\subsubsection{Verkürzung}
$$ l = k \; \cfrac{\lambda}{2} \quad n \in [0.93,0.97]$$

\subsection{Antennengewinn}
\subsubsection{zum Dipol}
\begin{align*}
&G_{D} = \cfrac{P_{V}}{P_{D}} \\[5pt]
&g_{d} = 10 \; log_{10}\left(\cfrac{P_{V}}{P_{D}}\right)dbd \\[5pt]
&g_{d} = 20 \; log_{10}\left(\cfrac{E_{V}}{E_{D}}\right)dbd \\[5pt]
\end{align*}

\subsubsection{zum isotropen Strahler}
\begin{align*}
&G_{i} = \cfrac{P_{V}}{P_{i}} \\[5pt]
&g_{i} = 10 \; log_{10}\left(\cfrac{P_{V}}{P_{i}}\right)dbd \\[5pt]
&g_{i} = 20 \; log_{10}\left(\cfrac{E_{V}}{E_{i}}\right)dbd \\[5pt]
\end{align*}
\subsubsection{ERP}

\begin{align*}
&P_{ERP} = \cfrac{P_{ERIP}}{1.64} \\[5pt]
&P_{ERP} = G_{D}\;P_{S} \\[5pt]
&P_{ERP} = P_{S} \; 10^{\cfrac{g_{d}}{10db}}  \\[5pt]
&P_{ERP} = G_{D}\; (P_{Sender}- P_{Verlust}) \\[5pt]
\end{align*}

\subsubsection{ERIP}

\begin{align*}
&P_{ERIP} = 1.64 \; P_{ERP}  \\[5pt]
&P_{ERIP} = G_{i}P_{S} \\[5pt]
&P_{ERIP} = P_{S} \; 10^{\cfrac{g_{i}}{10db}}  \\[5pt]
&P_{ERIP} = G_{i}\; (P_{Sender}- P_{Verlust}) \\[5pt]
\end{align*}

\subsubsection{Q-Match/$\cfrac{\lambda}{4}$ \;- Trafo}
$Z_{Kabel}=\sqrt{Z_{Ant}Z_{Leitung}}$

\newpage
\section{Leitungen}

\subsection{Wellenwiederstand}

$$Z_{w} = \sqrt{\cfrac{L'}{C'}}$$

\subsubsection{Paralleldrahtleitung}

$$Z_{w} = \cfrac{120\Omega}{\sqrt{\varepsilon_{r}}} \; ln\left(\cfrac{2a}{d}\right)$$

\subsubsection{Koaxialleitung}
$$Z_{w} = \cfrac{60\Omega}{\sqrt{\varepsilon_{r}}} \; ln\left(\cfrac{D}{d}\right)$$

\subsection{Verkürzungsfaktor}

\begin{align*}
&\nu = \cfrac{1}{\sqrt{L'C'}} \\[5pt]
&k = \cfrac{\nu}{c}  \\[5pt]
&k = \cfrac{1}{\sqrt{\varepsilon_{r}}}  \\[5pt]
\end{align*}

\subsection{Dämpfung}

$$n = \sqrt{\cfrac{f_{hoch}}{f_{niedrig}}}$$

\subsection{Transformationsleitung}
\begin{align*}
&R_{i} = Z_{w}=Z_{ant} \\[5pt]
&Z = \sqrt{Z_{1}Z_{2}}\\[5pt]
&l = (2n-1) \; \cfrac{\lambda}{4} \; k \\[5pt]
\end{align*}

\subsubsection{Koaxialleitung}
\begin{align*}
&Z=\cfrac{138\Omega}{\sqrt{\varepsilon_{r}}} \; \left( \cfrac{D}{d}\right) \\[5pt]
&D = d \; 10^{\cfrac{Z}{138\Omega}}\\[5pt]
\end{align*}

\newpage
\section{Signale}
\subsection{Effektivspannung}
\subsubsection{Sinus}
$$U_{eff}=\cfrac{\text{\^{U}}}{\sqrt{2}}$$
\subsubsection{Dreieck}
$$U_{eff}=\cfrac{\text{\^{U}}}{\sqrt{3}}$$
\subsubsection{Rechteck}
$$U_{eff}=\text{\^{U}}\sqrt{DutyCycle}$$

\subsection{Wellenlänge \& Frequenz}
\[
\begin{minipage}{.5\linewidth}
  $\lambda=\cfrac{c}{f}$ \\[5pt]
  $f=\cfrac{c}{\lambda}$ \\[5pt]
  $u = \^{u}\sin(\omega t+\varphi)$
\end{minipage}%
\begin{minipage}{.5\linewidth}
  \fontsize{8pt}{9pt}\selectfont
  $c = Lichtgeschwindugkeit \approx 3*10^{8}$ \\
  $c = 2.99792458*10^{8}$

\end{minipage}
\]

\subsection{Bandbreite}

\subsubsection{DSB}
$$b_{AM}=2f_{mod}$$

\subsubsection{SSB}
\begin{align*}
&b_{SSB} = f_{NFmax}-f_{NFmin} \\[5pt]
&b_{SSB} \approx f_{mod} \\[5pt]
\end{align*}

\subsubsection{FM}

\begin{align*}
&b_{FM} = 2(\Delta f_{T} + f_{mod}) \\[5pt]
&b_{FM} \approx 2 \; \Delta f_{T} \quad \quad f_{mod} \ll \Delta f_{T} \\[5pt]
&b_{FM} \approx 2 \; f_{mod} \quad \quad m < 0.5 \\[5pt]
\end{align*}

\subsubsection{CW}
$$b_{CW} = \cfrac{5*WPM}{1.2}$$

\subsubsection{RTTY}
$$b_{RTTY}=2\; \left(\cfrac{\Delta f}{2}+1.6 Bd\right)$$

\subsection{Modulationsindex FM}
$$m=\cfrac{\Delta f_{t}}{f_{mod}}$$

\subsection{Besselfunktion}
$$u = \^u_{0}\sin(\omega_{t}t-m\cos(\omega_{m}t))$$

\subsection{Peak Envelope Power}
\[
\begin{minipage}{.5\linewidth}
  $PEP = P_{c}(1+m)^2$ 
\end{minipage}%
\begin{minipage}{.5\linewidth}
  \fontsize{8pt}{9pt}\selectfont
  $PEP = Peak Envelope Power$ \\
  $P_{c} = Carrier-Power(Tr\ddot{a}gerleistung)$ \\
  $m = Modulationsgrad \; bei \; AM$ \\
\end{minipage}
\]

\newpage
\section{Modulation - Demodulation}
\subsection{Modulationsgrad}
$$m = \cfrac{\text{\^{U}}_{mod}}{\text{\^{U}}_{T}}$$

\newpage
\section{Frequenzaufbereitung}
\subsection{Überlagerung}
\subsubsection{$f_{osc} \;$>$\; f_{e}$}
\[
\begin{minipage}{.5\linewidth}
	
  $f_{z} = \cfrac{f_{sp}-f_{e}}{2}$ \\[5pt]
  $f_{osc} = f_{e} + f_{z}$
\end{minipage}%
\begin{minipage}{.5\linewidth}
  \fontsize{8pt}{9pt}\selectfont
  $f_{e} = Eingangsfrequenz$ \\
  $f_{osc} = Ueberlagerungsfrequenz$ \\
  $f_{z} = Zwischenfrequenz$ \\
  $f_{sp} = Spiegelfrequenz$ \\
\end{minipage}
\]
\subsubsection{$f_{osc} \;$<$\; f_{e}$}
\[
\begin{minipage}{.5\linewidth}
	
  $f_{z} =f_{e}- f_{osc}$ \\[5pt]
  $f_{sp} = f_{e} - 2 f_{z}$
\end{minipage}%
\begin{minipage}{.5\linewidth}
  \fontsize{8pt}{9pt}\selectfont
  $f_{e} = Eingangsfrequenz$ \\
  $f_{osc} = Ueberlagerungsfrequenz$ \\
  $f_{z} = Zwischenfrequenz$ \\
  $f_{sp} = Spiegelfrequenz$ \\
\end{minipage}
\]
\subsection{Frequenz 3.Ordnung}
$$ 2f_{1}-f_{2} \wedge 2f_{2}-f_{1}$$
\newpage
\section{Übertragungstechnik}
\subsection{Nquisttheorem}
$$f_{abt} > 2f_{i max}$$

\subsection{Dynamik}
$$D=20 \; log \left(\frac{U_{max}}{U_{min}}\right) dB$$

\subsection{Baudrate}
$$\nu_{u}=\cfrac{1}{t_{1bit}} \; Bd$$

\subsection{FSK}
\subsubsection{Bandbreite}
\begin{align*}
&b_{FSK} = 2(\Delta f_{T} + f_{mod}) \\[5pt]
&b_{FSK} \approx 2 \; \left(\cfrac{\Delta F}{2}+1.6 \; f_{u} \right)\\[5pt]
\end{align*}

\subsection{PSK}
\subsubsection{Bandbreite}
\begin{align*}
&b_{PSK} = 2(\Delta f_{T} + f_{mod}) \\[5pt]
&b_{PSK} = 2 \cfrac{\nu_{u}}{2} = \nu_{u} \\[5pt]
\end{align*}

\subsection{Totales Verbindungssystem}
\[
\begin{minipage}{.5\linewidth}
  $N=S\cfrac{S-1}{2}$
\end{minipage}%
\begin{minipage}{.5\linewidth}
  \fontsize{8pt}{9pt}\selectfont
  $N = Strecken$ \\
  $S = Stationen$

\end{minipage}
\]

\newpage
\section{Messtechnik}
\subsection{Wheatstonsche Messbrücke}

\begin{center}
\begin{circuitikz}[scale=0.75,transform shape] \draw
(3,5) to[ short, o-o ] (6,5)
(6,1) node[anchor=west] {$-$}
(6,5) node[anchor=west] {$+$}
(3,5) to[ R, l_=$R_1$, *-* ] (1,3)
(1,3) to[ R, l_=$R_x$, *-* ] (3,1)
(3,1) to[ R, l_=$R_4$, *-* ] (5,3)
(5,3) to[ R, l_=$R_3$, *-* ] (3,5)
(1,3) to[ ammeter, *-* ] (5,3)
(3,1) to[ short, *-o ] (6,1); 
\end{circuitikz}
\end{center}

$$R = \cfrac{R_{4}R_{1}}{R_{3}}$$

\subsection{Shunt}
\[
\begin{minipage}{.5\linewidth}
  $U=R_{Instr}I_{Instr} = R_{p}I_{P}$ \\[5pt]
  $I_{p}=I_{Messbereich}-I_{Instrument}$ \\[5pt]
  $R_{p}=\cfrac{U}{I_{p}}$\\[5pt]
  $R_{p}=\cfrac{R_{Instr}}{n-1}$\\[5pt]
  $R_{s}=R_{Instr}(n-1)$\\[5pt]
\end{minipage}%
\begin{minipage}{.5\linewidth}
  \fontsize{8pt}{9pt}\selectfont
  $R_{Instr} = Instrumentwiderstand$ \\
  $R_{p} = Shuntwiderstand \; parallel$ \\
  $R_{s} = Shuntwiderstand \; seriell$ \\
  $I_{p} = Strom \; durch \; Shunt$\\
  $I_{instr} = Instrumentenstrom$\\
  $n = Messbereichserweiterungsfraktor$
\end{minipage}
\]

\subsection{SWR/VSWR}
\[
\begin{minipage}{.5\linewidth}
  $s=\cfrac{U_{max}}{U_{min}} = \cfrac{U_{v}+U_{r}}{U_{v}-U_{r}} = \cfrac{1+|r|}{1-|r|}= \cfrac{\sqrt{P_{v}}+\sqrt{P_{r}}}{\sqrt{P_{v}}-\sqrt{P_{r}}}$ \\[5pt]
  $|r| =\cfrac{U_{r}}{U_{v}} = \sqrt{\cfrac{P_{r}}{P_{v}}} = \cfrac{s-1}{s+1}$ \\[5pt]
  $s =\cfrac{R_{2}}{Z} \quad R_{2} \geq Z$ \\[5pt]
  $s =\cfrac{Z}{R_{2}} \quad R_{2} \leq Z$ \\[5pt]
  
\end{minipage}%
\begin{minipage}{.5\linewidth}
  \fontsize{8pt}{9pt}\selectfont
  $s = SWR/VSWR$ \\
  $r = Reflexionsfaktor$ \\
  $Z = Wellenwiederstand \; (der \; Leitung)$ \\
  $R_{2} = Abschlusswiederstand$ \\
  $U_{v} = hinlaufende \; Welle$ \\
  $U_{r} = rücklaufende \; Welle$ \\
  
\end{minipage}
\]

\newpage
\section{Gerätetechnik}
\subsection{Empfindlichkeit}
\[
\begin{minipage}{.5\linewidth}
  $P_{R}=kT_{0} bF$ \\[5pt]
  $U_{R}=\sqrt{kT_{0} bRF}$
  
\end{minipage}%
\begin{minipage}{.5\linewidth}
  \fontsize{8pt}{9pt}\selectfont
  $k = 1.38*10^{-23} \; (Boltzmann \; Konstante)$ \\
  $T_{0} = Temperatur \; [K]$ \\
  $b = Bandbreite \; [Hz]$ \\
  $R = Eingangswiederstand$ \\
  $F = Rauschfaktor$ \\
  $P_{R} = Rauschleistung$ \\
  $U_{R} = Rauschspannung$ \\
  
\end{minipage}
\]

\newpage
\section{EMV und Sicherheit}
\subsection{Windlast}
\[
\begin{minipage}{.5\linewidth}
  $F_{A}=pA$ \\[5pt]
  
\end{minipage}%
\begin{minipage}{.5\linewidth}
  \fontsize{8pt}{9pt}\selectfont
  $p = Staudruck \; \left[\cfrac{N}{m^2}\right]$ \\
  $A = Wirckfläche \; [m^2]$ \\
  
\end{minipage}
\]

\subsection{Biegemoment}
$$M_{A} = \sum F_{i} l_{i}$$

\subsection{Sicherheitsabstand}
$$d = \cfrac{\sqrt{30 \Omega \; P_{ERIP}}}{E}$$
\end{document}
